\section*{C\-P\-R\-O\-J}

{\itshape projet de renderer consistant a analyser des donnees de openstreetmap pour en faire un rendu graphique en language C utilisant S\-D\-L et Libxml2}



-\/\-Mini\-O\-S\-M a maintenant un site web ! A l'adresse \-: \href{http://cloudstrif.github.io/}{\tt http\-://cloudstrif.\-github.\-io/}

-\/\-La compilation se fait avec -\/\-Wall

-\/\-Une presentation dans /source/site\-Web/index.html

-\/\-Le programme a plusieurs options mais sans argument, il lance le produit final.

-\/\-Les touches /!\textbackslash{} \-:Q\-S\-D\-Z pour se deplacer. \char`\"{}fleche haut\char`\"{}= zoom avant \char`\"{} fleche bas\char`\"{}=zoom arrière \char`\"{}fleche droite\char`\"{}=inclinaison \char`\"{}fleche gauche\char`\"{}=declinaison \char`\"{}o\char`\"{}=centrage de map (si perdu dans la map) \char`\"{}p\char`\"{}=debut d'affichage des balises names \char`\"{}m\char`\"{}= ne plus afficher les balises names. b=inverser n=deverser

-\/\-L'option --zoom permet de se déplacer dans la map et on peut zoomer en appuyant sur la touche a ou z pour le dezoom Les flèches permettent de bouger dans la map (A U\-T\-I\-L\-I\-S\-E\-R L\-O\-R\-S\-Q\-U\-E L\-A M\-A\-P E\-S\-T T\-R\-O\-P G\-R\-A\-N\-D\-E (manuellement pour le moment)

-\/(A\-P\-I O\-V\-E\-R\-P\-A\-S\-S)vous pouvez placer 4 arguments (attributs de $<$bounds$>$) et cela trace la map.

-\/\-A chaque fin de programme, celui-\/ci génère un fichier sortie.\-bmp.

-\/style.\-xml est un fichier qui permet de changer les couleurs des balises grace au systeme R\-G\-B\-A

-\/l'option --svg génère un fichier dessin.\-svg.\-Faire make svg à chaque fois que vous voulez le faire /!\textbackslash{}

-\/l'option --recording name permet d'enregistrer une video des déplacements divers effectués

-\/l'option --play name permt de jouer ce que vous avez enregistré via la commande recording.

-\/l'option --search permet de localiser un lieux \-: ./projet ../utils/examples/12\-\_\-paris\-\_\-pitie\-\_\-salpetriere.osm \char`\"{}\-Place Louis Armstrong\char`\"{} --search



-\/\-L'option -\/past trace une map façon carte au trésor , les calculs ont ete effectués de sorte à ce que la terre soit plate\-: 

-\/\-L'option -\/present est un rendu utilisant les projections car la terre est sphérique\-: 

L'option -\/futur (lourd..) donne une vue en perspective de la carte\-: 

Paris one day ..... 

-\/make meteo génère meteo , puis ./meteo lat lon -\/$>$ affiche la météo dans le terminal 